\chapter{Buffer Overflow}

\section{Definición y explicación}
\noindent El \textbf{buffer overflow} o desbordamiento de búfer es un fenómeno que ocurre cuando un programa intenta almacenar más datos en un bloque de memoria (\textit{buffer}) del que este puede contener. Los búferes son segmentos de memoria reservados específicamente para almacenar datos temporalmente. Estos datos pueden variar ampliamente, desde entradas del usuario, datos recibidos de un archivo, hasta paquetes de información que forman parte de comunicaciones en red. Sin embargo, cuando los datos que se intentan almacenar sobrepasan la capacidad asignada al búfer, se produce un desbordamiento. En esta situación, los datos excedentes se expanden hacia otras zonas de la memoria, sobrescribiendo potencialmente información crítica que el programa no tenía intención de modificar, lo cual puede alterar su funcionamiento.

\vspace{0.3cm}
\noindent Este tipo de vulnerabilidad es particularmente común en lenguajes de programación como \texttt{C} o \texttt{C++}, donde la gestión de la memoria no es controlada de manera automática por el lenguaje. En estos lenguajes, la memoria asignada a cada variable es administrada manualmente por el programador o, de manera predeterminada, sin verificación de límites. Cuando se intenta escribir más información en un búfer del que este puede albergar, los datos adicionales pueden sobrescribir variables adyacentes, estructuras de control del programa, o incluso direcciones de retorno en la pila de memoria. En el peor de los casos, esto puede hacer que el programa falle de manera catastrófica, se comporte de manera errática, o, en los casos más graves, ejecute código malicioso que haya sido inyectado por un atacante. Esto último es particularmente peligroso cuando el desbordamiento afecta regiones de memoria sensibles o protegidas del sistema, que el programa no tiene autorización para modificar.

\vspace{0.3cm}
\noindent Los buffer overflows no siempre son fáciles de detectar, ya que pueden no causar problemas visibles durante la ejecución normal del software, y es posible que pasen desapercibidos por largos periodos de tiempo. Sin embargo, la ausencia de síntomas evidentes no implica que el problema sea menor o inofensivo. Un desbordamiento puede generar fallos importantes en el programa y en la integridad de los datos, especialmente en aplicaciones que manejan grandes volúmenes de datos o que se ejecutan durante largos periodos de tiempo. Estos fallos pueden comprometer la fiabilidad, seguridad y usabilidad del software.

\newpage
\subsection{Por qué es un problema grave}
\noindent 
El \textbf{desbordamiento de búfer} es considerado una vulnerabilidad grave en el ámbito de la seguridad informática, y no necesariamente debe ser explotado por un atacante para que represente un riesgo significativo. A menudo, los desbordamientos de búfer pueden estar presentes en un sistema durante mucho tiempo sin ser detectados, y sin provocar consecuencias inmediatas. Esto no significa que su presencia sea inofensiva; de hecho, un desbordamiento de búfer puede causar problemas muy serios en el funcionamiento de un programa en diversas formas.

\vspace{0.3cm}
\noindent Un aspecto importante a tener en cuenta es que un desbordamiento de búfer puede corromper los datos almacenados en áreas de la memoria que son esenciales para el funcionamiento adecuado del software. Este tipo de corrupción de datos puede llevar a \textbf{fallos aleatorios} en el programa, los cuales son notoriamente difíciles de depurar. Estos fallos suelen manifestarse solo en circunstancias específicas, como cuando se introducen valores de entrada que exceden el tamaño permitido por el búfer. Esto da lugar a lo que se conoce como \textbf{errores intermitentes}, que son complicados de diagnosticar debido a que no se producen de forma constante, y no siempre es sencillo reproducir las condiciones que los desencadenan.

\vspace{0.3cm}
\noindent Un desbordamiento de búfer también puede \textbf{interrumpir el flujo de ejecución del programa}, provocando bloqueos, pérdida de datos o respuestas inesperadas. Esto es problemático no solo porque afecta la experiencia del usuario, sino porque compromete la estabilidad general del sistema en el que el programa se está ejecutando. Un programa que falla ocasionalmente en ciertas circunstancias puede afectar a otros procesos y servicios que dependan de él, generando una \textbf{cascada de fallos} en otros componentes del sistema. Esto puede llevar a un colapso total en aplicaciones de misión crítica, donde la confiabilidad y la estabilidad son esenciales para el funcionamiento del sistema en su conjunto.

\vspace{0.3cm}
\noindent Más allá de los problemas de integridad de datos y estabilidad, el desbordamiento de búfer presenta un \textbf{riesgo significativo de seguridad}. En situaciones donde un atacante puede controlar los datos que se ingresan en un búfer, puede aprovechar el desbordamiento para modificar la memoria del sistema de forma deliberada metiendo shellcode. En algunos casos, el atacante podría alterar el flujo de ejecución del programa para ejecutar código arbitrario(shellcode), dándole la posibilidad de tomar el control total del sistema en el que el programa vulnerable se está ejecutando. Por esta razón, los desbordamientos de búfer son uno de los vectores de ataque más comunes en exploits y ataques de escalación de privilegios.

\vspace{0.3cm}
