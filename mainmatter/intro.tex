\chapter*{INTRODUCCIÓN}
\addcontentsline{toc}{section}{\textbf{INTRODUCCIÓN}}

\noindent
En esta práctica, nos adentraremos en el estudio y solución de un problema crítico en la programación: el desbordamiento de búfer (buffer overflow). Este fenómeno ocurre cuando un programa escribe más datos en un área de memoria de lo que estaba previsto, lo que puede causar fallos graves, desde comportamientos inesperados hasta vulnerabilidades de seguridad que los atacantes pueden explotar. Durante el desarrollo de esta práctica, aprenderemos a identificar este tipo de errores en el código y a corregirlos de manera eficiente, utilizando diversas herramientas de desarrollo y depuración.
\vspace{0.3cm}
\\
El propósito de esta práctica es, no solo entender qué es un desbordamiento de búfer, sino también adquirir las habilidades necesarias para detectarlo en código real. Herramientas como Valgrind serán clave para analizar la memoria utilizada por el programa y detectar problemas. Como editor de texto usaré Kate debido a preferencia personal y exploraremos otras herramientas útiles que facilitan el proceso de depuración.
\vspace{0.3cm}
\\
Cabe destacar que el ingeniero, por naturaleza, busca optimizar sus procesos y resultados, minimizando el tiempo de trabajo y maximizando la eficiencia. A lo largo de esta práctica, descubriremos varias estrategias y herramientas que nos permitirán cumplir con este objetivo. Estas herramientas no solo mejorarán nuestro flujo de trabajo, sino que también permitirán que el código sea más robusto, seguro y eficiente. \par
